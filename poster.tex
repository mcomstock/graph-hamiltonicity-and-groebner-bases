%% Use the hmcposter class with the thesis document-class option.
\documentclass[thesis]{hmcposter}
\usepackage{graphicx}
\usepackage{natbib}
\usepackage{booktabs}
\usepackage{subfig}
\usepackage{amsmath}
\usepackage{textcomp}
\usepackage{url}
\usepackage{tikz}
\usetikzlibrary{arrows}
\usetikzlibrary{decorations.markings}

\tikzstyle{every node}=[circle, draw, fill=black!50, inner sep=0pt, minimum width=4pt]
\tikzset{->-/.style={decoration={
      markings,
      mark=at position .5 with {\arrow{>}}},postaction={decorate}}}
\tikzset{-<-/.style={decoration={
      markings,
      mark=at position .5 with {\arrow{<}}},postaction={decorate}}}

%\renewenvironment{comment}{}{}

\newcommand{\cone}{\mathbf{cone}}
\newcommand{\fm}{{\mathfrak m}}
\newcommand{\R}{{\mathbb R}}
\newcommand{\Z}{\mathbb{Z}}
\newcommand{\N}{\mathbb{N}}
\newcommand{\G}{\mathbb{G}}
\newcommand{\C}{\mathbb{C}}
\newcommand{\F}{\mathbb{F}}
\newcommand{\PP}{\mathbb{P}}
\newcommand{\rx}{\mathbb{R}[x_1,\ldots,x_n]}
\newcommand{\gen}[1]{\langle #1 \rangle_{\F_2}}
\newcommand{\FC}{{\overline{\mathbb{F}}}}
\newcommand{\kS}{\k^\infty}
\newcommand{\kD}{\kc^*}
\newcommand{\Q}{\mathbb{Q}}
\newcommand{\V}[1]{V_\kc(#1)}
\newcommand{\W}{\mathcal{W}}
\newcommand{\M}{\mathcal{M}}
\newcommand{\x}{{\bf x}}
\newcommand{\RD}{{R^*}}
\newcommand{\codim}[2]{dim(#1/#2)}
\newcommand{\Var}[2]{{\mathcal{V}_{#1}(#2)}}
\newcommand{\norm}[1]{\left\Vert#1\right\Vert}
\newcommand{\abs}[1]{\left\vert#1\right\vert}
\newcommand{\set}[1]{\left\{#1\right\}} \newcommand{\Real}{\mathbb R}
\newcommand{\eps}{\varepsilon} \newcommand{\To}{\longrightarrow}
\newcommand{\BX}{\mathbf{B}(X)} \newcommand{\A}{\mathcal{A}}

\renewcommand{\(}{\left(}
\renewcommand{\)}{\right)}
\renewcommand{\iff}{if and only if\xspace}
\newcommand{\<}{\langle}
\renewcommand{\>}{\rangle}

\newcommand{\genR}[1]{\langle #1 \rangle_R}
\newcommand{\genK}[1]{\langle #1 \rangle_\k}
\newcommand{\ZZ}[1]{\mathbb{Z}/#1\mathbb{Z}}
\renewcommand{\k}{\mathbb K}
\newcommand{\kc}{{\overline{\k}}}
\newcommand{\g}{\mathcal{G}}
\newcommand{\ideal}[1]{\langle #1 \rangle}

\newcommand{\aln}[1]{\begin{align*} #1 \end{align*}} %fast align
\newcommand{\fitp}[1]{\left( #1 \right)} %fit parentheses
\newcommand{\ds}[1]{$ \displaystyle #1 $}

%% Author of the thesis.
\author{Max Comstock}

%% The year of your thesis poster's creation.
\posteryear{2014}

%% Thesis Title.
\title{Graph Hamiltonicity\\ and Gr\"obner Bases}

%% The name of the class for which the poster was created.
%% Generally we see posters for thesis and Clinic, but sometimes
%% other classes require or allow the creation of posters to
%% communicate the results of a project.
%% 
%% Use the format Math nnn: Class Title.
\class{Summer 2014}

%% Advisor(s) name or names.  Separate with \and.
\advisor{Professor Omar}

%% Reader(s) name or names.  Separate with \and.
%\reader{Melissa O'Neill \and Charlie Watts}



\pagestyle{fancy}

\begin{document}

\begin{poster}

\section{Introduction}
% Note that we're not labeling sections because you shouldn't be
% doing a lot of referring back and forth in your poster---let the
% interested folks read your thesis or Clinic report, or ask
% questions.

The purpose of this project is to examine an algebraic method used to find Hamiltonian cycles in graphs, by creating a system of polynomial equations that has solutions corresponding to cycles in the graph. The ideal generated from these polynomials is guaranteed to have certain properties depending on the number of Hamiltonian cycles in the graph. The results include a summary of existing theorems about these ideals, along with a few discoveries about the form of their reduced Gr\"obner bases.


\section{Gr\"obner Bases}

A central object in our study is Gr\"{o}bner bases.  Given an ideal $I$ of $R$ and a term order $\prec$, a finite subset $\g$ of $I$ is a \emph{Gr\"{o}bner basis} with respect to $\prec$ if the ideal
\[
in_{\prec}(I) = \ideal{in_{\prec}(f) : f \in I},
\]
is generated by the initial terms of $\g$.

The ideal $in_{\prec}(I)$ is called the \emph{initial ideal} of $I$ with respect to $\prec$.  The Gr\"{o}bner basis $\g$ is \emph{minimal} if no leading term of $f \in \g$ divides any other leading term of polynomials in $\g$.

The Gr\"obner basis $\g$ is \emph{reduced} if no leading term of $f \in \g$ divides any monomial in any other polynomials in $\g$. If $I \neq \{0\}$ is a polynomial ideal, then $I$ has a unique reduced Gr\"obner basis for any given monomial ordering.

From \cite{coxlittleoshea}.

\section{Graph Ideals}

Let $G = (V,A)$ be a simple directed graph on vertices $V = \{1, \ldots, n\}$. Assume that the characteristic of $\k$ is relatively prime to $n$ and that $\omega \in \k$ is a primitive $n$-th root of unity. Consider the following system in $\k[x_1, \ldots, x_n]$:
	\begin{align*}
		H_G = \{x_i^n - 1 = 0, \prod_{j \in \delta^+(i)} (\omega x_i - x_j) = 0 \, : \, i \in V\}.
	\end{align*}
Here, $\delta^+(i)$ denotes those vertices $j$ which are connected to $i$ by an arc going from $i$ to $j$ in $G$. The system $H$ has a solution over $\kc$ if and only if $G$ has a Hamiltonian cycle.

From \cite{deloera10}.

\section{Procedure}%

To detect whether a graph has any Hamiltonian cycles, we can use the following procedure given a graph $G$:
\begin{itemize}
	\item Calculate the graph ideal $H_G$ for $G$.\\
	\item Use software to find a reduced Gr\"obner basis for $H_G$.\\
	\item If the Gr\"obner basis is not $\{1\}$, the graph has a Hamiltonian cycle.
	\item We want to learn as much as we can about the structure of this Gr\"obner basis and how it relates to the structure of the graph.
\end{itemize}

\section{For Further Information}

\begin{itemize}
\item E-mail address: \url{mcomstock@g.hmc.edu}.
\item More links to come.
\end{itemize}


\vfill
\columnbreak


\section{Unique Hamiltonicity}

We know that a directed graph has a unique Hamiltonian cycle if a Gr\"obner basis of $H_G$ has the form $H_G = \<g_1,\ldots,g_k\>$, where
\begin{align*}
	g_i = \left \{
		\begin{array}{ll}
		x_{v_{k-i}} - \omega^{k-1} x_{v_k} & i = 1,\ldots,k-1\\
		x_{v_k}^k - 1 & i = k.
		\end{array} \right .
\end{align*}
Proof can be found in \cite{deloera10}.

\section{New Discoveries}

\begin{itemize}
	\item Let $G$ be a graph with $n$ vertices and one or more Hamiltonian cycles, and $k$ be a natural number (not including zero). Then we find, for each $i$ such that $1 \leq i \leq n$, that $\mathrm{LT}(g) = x_i^k$ for some $g$ in the reduced Gr\"obner basis of $H_G$.\\
	\item For a graph $G$ with $n$ vertices and one or more Hamiltonian cycles, we find $x_n^n - 1$ is in the reduced Gr\"obner basis of $H_G$. Furthermore, this is the only polynomial in the reduced Gr\"obner basis that contains $x_n$ and no other variables.
\end{itemize}

\section{Sample Calculation}

Graph:
\begin{center}
\begin{tikzpicture}[thick, scale=2]
	\path (4,0)++(45:4) coordinate(upper);
	\path (4,0)++(-45:4) coordinate(lower);
	\draw[->-] (upper) to (0,0);
	\draw[->-] (lower) to (0,0);
	\draw[->-] (upper) to[bend right] (lower);
	\draw[->-] (lower) to[bend right] (upper);
	\draw[->-] (0,0) node{} to (4,0);
	\draw[->-] (4,0) to ++(45:4) node{};
	\draw[->-] (4,0) node{} to ++(-45:4) node{};
\end{tikzpicture}
\end{center}

Reduced Gr\"obner basis: \ds{\{x_4^4-1, x_3^2+x_4^2, 2x_2+(z+1)x_3+(z+1)x_4, 2x_1+(-z+1)x_3+(-z+1)x_4\}}.


\begin{thebibliography}{9}

\bibitem{coxlittleoshea}
	Cox, Little, O'Shea,
	\emph{Ideals, Varieties, and Algorithms}.
	Springer, New York,
	Third Edition,
	2007.

\bibitem{deloera07}
	J.A. De Loera, J. Lee, S. Marulies, S. Onn,
	\emph{Expressing Combinatorial Optimization Problems by Systems of Polynomial Equations and the Nullstellensatz}.
	Unpublished.

\bibitem{deloera10}
	J.A De Loera, C. Hillar, P.N. Malkin, M. Omar,
	\emph{Recognizing Graph Theoretic Properties with Polynomial Ideals}.
	Unpublished.

\bibitem{hillar06}
	C. Hillar, T. Windfeldt,
	\emph{Algebraic Characterization of Uniquely Vertex Colorable Graphs}.
	Unpublished.

\end{thebibliography}

\vfill

\section{Acknowledgments}

I would like to thank Professors Omar and Ruiz for all of their guidance, as well as the members of their research teams for their support and feedback.

\vfill
\end{poster}

\end{document}

 
