\documentclass{beamer}
\usepackage{amsthm}
\usepackage{amsmath,amssymb}
\usepackage{verbatim}

\newtheorem*{conj}{Conjecture}
\newtheorem*{prop}{Proposition}
\newtheorem*{defin}{Definition}

\newcommand{\aln}[1]{\begin{align*} #1 \end{align*}} % fast align

\title{Week 3 Presentation}
\author{Max Comstock}
\date{Summer 2014}

\begin{document}

\frame{\titlepage}


\begin{frame}
\frametitle{How to represent a primitive $n$th root of unity}
\begin{defin}
  An $n$th root of unity is \textbf{primitive} if it is not a $k$th root of unity for some smaller $k$:
  \aln{
    z^k \neq 1 \quad (k = 1, 2, \ldots, n-1).
  }
\end{defin}
\end{frame}


\begin{frame}
\frametitle{Cyclotomic poynomial $\Phi_n$}
\begin{defin}
  The \textbf{$n$th cyclotomic polynomial} is the unique irreducible polynomial with integer coefficients whose roots are the $n$th primitive roots of unity.
\end{defin}
Examples:
\aln{
  \Phi_3(z) &= z^2 + z + 1\\
  \Phi_4(z) &= z^2 + 1\\
  \Phi_5(z) &= z^4 + z^3 + z^2 + z + 1\\
  \Phi_6(z) &= z^2 - z + 1
}
\end{frame}


\begin{frame}
\frametitle{Using $\Phi_n$ to create a field with $n$th roots of unity}
We will begin with the ring
Our goal is to create the ring
\aln{
  \mathbb{K}[x_1, \ldots, x_n]
}
where $\mathbb{K}$ is a field containing the primitive $n$th roots of unity. We can create such a field as follows:
\aln{
  \mathbb{K} = \frac{\mathbb{Q}[z]}{\Phi_n(z)}.
}
Using the relationship
\aln{
  \left( \frac{\mathbb{Q}[z]}{\Phi_n(z)} \right)[x_1, \ldots, x_n] \cong \frac{\mathbb{Q}[x_1, \ldots, x_n, z]}{\Phi_n},
}
We can choose
\aln{
  \mathbb{K}[x_1, \ldots, x_n] = \frac{\mathbb{Q}[x_1, \ldots, x_n, z]}{\Phi_n(z)}.
}
\end{frame}


\begin{frame}[fragile]
\frametitle{Implementation using Macaulay2}
\begin{center}
  \begin{verbatim}
    R = QQ[x_1, x_2, x_3, x_4, x_5, x_6, z];
    CYCLOTOMIC_POLY = z^2 - z + 1;
    R = R/CYCLOTOMIC_POLY;
    IDEAL_GEN = {x_1^6 - 1, x_2^6 - 1, x_3^6 - 1,
             x_4^6 - 1, x_5^6 - 1, x_6^6 - 1,
	     z*x_1 - x_2,
	     z*x_2 - x_3,
	     z*x_3 - x_4,
	     z*x_4 - x_5,
	     z*x_5 - x_6,
	     z*x_6 - x_1};
    idealOfGraph = ideal IDEAL_GEN;
    graphBasis = flatten entries gens gb idealOfGraph;
    print toString graphBasis;
  \end{verbatim}
\end{center}
\end{frame}



\end{document}
