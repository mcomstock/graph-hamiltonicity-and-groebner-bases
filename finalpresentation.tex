\documentclass{beamer}
\usepackage{beamerthemesplit}
\usepackage{amsmath,amssymb}
\usepackage{amsthm}
\usepackage{graphicx}
\usepackage{lmodern}
\usepackage[T1]{fontenc}
\usepackage{color}
\usepackage{tikz}
\usetikzlibrary{decorations.markings}

\tikzstyle{every node}=[circle, draw, fill=black!50, inner sep=0pt, minimum width=4pt]
\tikzset{->-/.style={decoration={
      markings,
      mark=at position .5 with {\arrow{>}}},postaction={decorate}}}
\tikzset{-<-/.style={decoration={
      markings,
      mark=at position .5 with {\arrow{<}}},postaction={decorate}}}

\newtheorem*{conj}{Conjecture}
\newtheorem*{prop}{Proposition}
\newtheorem*{defin}{Definition}
\newtheorem*{thm}{Theorem}

\newcommand{\aln}[1]{\begin{align*} #1 \end{align*}} % fast align

\title{Graph Hamiltonicity and Gr\"obner Bases}
\author{Max Comstock}
\date{Summer 2014}

\begin{document}


\frame{\titlepage}

\section{Introduction}

\subsection{Algebraic Techniques}

\begin{frame}
\frametitle{Gr\"obner bases}
\begin{defin}
The \textbf{initial term} of a polynomial $f \in R$ with respect to $\prec$, denoted $in_{\prec}(f)$, is the largest monomial in $f$ with respect to $\prec$.
\end{defin}
\begin{defin}
Given an ideal $I$ of $R$ and a term order $\prec$, a finite subset $\mathcal{G}$ of $I$ is a \textbf{Gr\"{o}bner basis} with respect to $\prec$ if the ideal
\aln{
in_{\prec}(I) = \langle in_{\prec}(f) : f \in I \rangle,
}
is generated by the initial terms of $\mathcal{G}$.
\end{defin}
\end{frame}

\begin{frame}
\frametitle{A way to algebraically encode Hamiltonian cycles}
\begin{prop}
  Let $G = (V,A)$ be a simple directed graph on vertices $V = \{1, \ldots, n\}$. Assume that the characteristic of $\mathbb{K}$ is relatively prime to $n$ and that $z \in \mathbb{K}$ is a primitive $n$-th root of unity. Consider the following system in $\mathbb{K}[x_1, \ldots, x_n]$:
  \begin{align*}
    H_G = \{x_i^n - 1 = 0, \prod_{j \in \delta^+(i)} (z x_i - x_j) = 0 \, : \, i \in V\}
  \end{align*}
  Here, $\delta^+(i)$ denotes those vertices $j$ which are connected to $i$ by the arc going from $i$ to $j$ in $G$. The system $H$ has a solution over $\mathbb{K}$ if and only if $G$ has a Hamiltonian cycle.
\end{prop}
Source: ``Recognizing Graph Theoretic Properties with Polynomial Ideals'' by J.A. De Loera, C. Hillar, P.N. Malkin, and M. Omar.
\end{frame}

\begin{frame}
\frametitle{A way to algebraically encode Hamiltonian cycles}
\begin{prop}
  Let $G = (V,A)$ be a simple directed graph on vertices $V = \{1, \ldots, n\}$. Assume that the characteristic of $\mathbb{K}$ is relatively prime to $n$ and that {\color{red} $z \in \mathbb{K}$ is a primitive $n$-th root of unity}. Consider the following system in $\mathbb{K}[x_1, \ldots, x_n]$:
  \begin{align*}
    H_G = \{x_i^n - 1 = 0, \prod_{j \in \delta^+(i)} (z x_i - x_j) = 0 \, : \, i \in V\}
  \end{align*}
  Here, $\delta^+(i)$ denotes those vertices $j$ which are connected to $i$ by the arc going from $i$ to $j$ in $G$. The system $H$ has a solution over $\mathbb{K}$ if and only if $G$ has a Hamiltonian cycle.
\end{prop}
Source: ``Recognizing Graph Theoretic Properties with Polynomial Ideals'' by J.A. De Loera, C. Hillar, P.N. Malkin, and M. Omar.
\end{frame}

\begin{frame}
\frametitle{A way to algebraically encode Hamiltonian cycles}
\begin{prop}
  Let $G = (V,A)$ be a simple directed graph on vertices $V = \{1, \ldots, n\}$. Assume that the characteristic of $\mathbb{K}$ is relatively prime to $n$ and that $z \in \mathbb{K}$ is a primitive $n$-th root of unity. Consider the following system in $\mathbb{K}[x_1, \ldots, x_n]$:
  \begin{align*}
    H_G = \{{\color{red} x_i^n - 1 = 0}, \prod_{j \in \delta^+(i)} (z x_i - x_j) = 0 \, : \, i \in V\}
  \end{align*}
  Here, $\delta^+(i)$ denotes those vertices $j$ which are connected to $i$ by the arc going from $i$ to $j$ in $G$. The system $H$ has a solution over $\mathbb{K}$ if and only if $G$ has a Hamiltonian cycle.
\end{prop}
Source: ``Recognizing Graph Theoretic Properties with Polynomial Ideals'' by J.A. De Loera, C. Hillar, P.N. Malkin, and M. Omar.
\end{frame}

\begin{frame}
\frametitle{A way to algebraically encode Hamiltonian cycles}
\begin{prop}
  Let $G = (V,A)$ be a simple directed graph on vertices $V = \{1, \ldots, n\}$. Assume that the characteristic of $\mathbb{K}$ is relatively prime to $n$ and that $z \in \mathbb{K}$ is a primitive $n$-th root of unity. Consider the following system in $\mathbb{K}[x_1, \ldots, x_n]$:
  \begin{align*}
    H_G = \{x_i^n - 1 = 0, {\color{red} \prod_{j \in \delta^+(i)} (z x_i - x_j) = 0} \, : \, i \in V\}
  \end{align*}
  Here, $\delta^+(i)$ denotes those vertices $j$ which are connected to $i$ by the arc going from $i$ to $j$ in $G$. The system $H$ has a solution over $\mathbb{K}$ if and only if $G$ has a Hamiltonian cycle.
\end{prop}
Source: ``Recognizing Graph Theoretic Properties with Polynomial Ideals'' by J.A. De Loera, C. Hillar, P.N. Malkin, and M. Omar.
\end{frame}

\begin{frame}
\frametitle{Simple case: directed graphs}
\begin{defin}
  Let $z$ be a fixed primitive $k$-th root of unity. If $C$ is a directed cycle of length $k$ in a directed graph, with vertex set $\{v_1, \ldots, v_k\}$, the \textbf{cycle encoding} of $C$ is the following set of $k$ polynomials:
  \begin{align*}
    g_i = \left \{ \begin{matrix} x_{v_{k-i}} - z^{k-i} x_{v_k} & i = 1, \ldots, k-1\\ x_{v_k}^k - 1 & i = k \end{matrix} \right ..
  \end{align*}
\end{defin}
Note: define $H_{G,C} = \langle g_1,\ldots, g_i \rangle$. The $g_i$'s form a reduced Gr\"obner basis (which must be unique) for $H_{G,C}$.
\end{frame}

\begin{frame}
\frametitle{Simple case: directed graphs}
\begin{defin}
  Let $z$ be a fixed primitive $k$-th root of unity. If $C$ is a directed cycle of length $k$ in a directed graph, with vertex set $\{v_1, \ldots, v_k\}$, the \textbf{cycle encoding} of $C$ is the following set of $k$ polynomials:
  \begin{align*}
    g_i = \left \{ \begin{matrix} {\color{red} x_{v_{k-i}} - z^{k-i} x_{v_k}} & {\color{red} i = 1, \ldots, k-1}\\ x_{v_k}^k - 1 & i = k \end{matrix} \right ..
  \end{align*}
\end{defin}
Note: define $H_{G,C} = \langle g_1,\ldots, g_i \rangle$. The $g_i$'s form a reduced Gr\"obner basis (which must be unique) for $H_{G,C}$.
\end{frame}

\begin{frame}
\frametitle{Simple case: directed graphs}
\begin{defin}
  Let $z$ be a fixed primitive $k$-th root of unity. If $C$ is a directed cycle of length $k$ in a directed graph, with vertex set $\{v_1, \ldots, v_k\}$, the \textbf{cycle encoding} of $C$ is the following set of $k$ polynomials:
  \begin{align*}
    g_i = \left \{ \begin{matrix} x_{v_{k-i}} - z^{k-i} x_{v_k} & i = 1, \ldots, k-1\\ {\color{red} x_{v_k}^k - 1} & {\color{red} i = k} \end{matrix} \right ..
  \end{align*}
\end{defin}
Note: define $H_{G,C} = \langle g_1,\ldots, g_i \rangle$. The $g_i$'s form a reduced Gr\"obner basis (which must be unique) for $H_{G,C}$.
\end{frame}

\subsection{Example}

\begin{frame}
\frametitle{Simple example}
\begin{center}
	\begin{tikzpicture}[thick,scale=0.3]
		\draw[->-] (108:5) -- (5,0);
		\draw[-<-] (5,0) ++(72:5) -- (108:5);
		
		\draw[-<-] (0,0) -- (5,0);
		\draw[-<-] (5,0) node{} -- ++(72:5);
		\draw[-<-] (5,0)++(72:5) node{} -- ++(2*72:5);
		\draw[-<-] (5,0)++(72:5)++(2*72:5) node{} -- ++(3*72:5);
		\draw[-<-] (5,0)++(72:5)++(2*72:5)++(3*72:5) node{} -- (0,0) node{};
	\end{tikzpicture}
\end{center}
Let $z$ be a primitive 5\textsuperscript{th} root of unity. We are looking for solutions to the system $H$:
\begin{align*}
  x_i^5 - 1 &= 0 \quad 1 \leq i \leq 5\\
  (z x_1 - x_2) (z x_1 - x_3) (z x_1 - x_4) &= 0\\
  z x_2 - x_3 &= 0\\
  z x_3 - x_4 &= 0\\
  z x_4 - x_5 &= 0\\
  z x_5 - x_1 &= 0
\end{align*}
\end{frame}

\begin{frame}
\frametitle{Simple example (continued)}
Process for a single Hamiltonian cycle:
\begin{itemize}
\item Let $H_G$ be the ideal generated by the polynomials in the system $H$.
\item If we find a Gr\"obner basis for $H_G$ with respect to the ordering $x_5 < x_4 < x_3 < x_2 < x_1$, we find it is a generating set for $H_{G,C}$.
\end{itemize}
In our example, the reduced Gr\"obner basis is
\begin{align*}
  \{x_5^5 - 1, x_4 - x_5 z^4, x_3 - x_5 z^3, x_2 - x_5 z^2, x_1 - x_5 z\}.
\end{align*}
\end{frame}

\section{Computation and Roots of Unity}

\subsection{Definition}

\begin{frame}
\frametitle{How to represent a primitive $n$th root of unity}
\begin{defin}
  An $n$th root of unity is \textbf{primitive} if it is not a $k$th root of unity for some smaller $k$:
  \aln{
    z^k \neq 1 \quad (k = 1, 2, \ldots, n-1).
  }
\end{defin}
\end{frame}

\subsection{Algebraic Methods}

\begin{frame}
\frametitle{Cyclotomic poynomial $\Phi_n$}
\begin{defin}
  The \textbf{$n$th cyclotomic polynomial} is the unique irreducible polynomial with integer coefficients whose roots are the $n$th primitive roots of unity.
\end{defin}
Examples:
\aln{
  \Phi_3(z) &= z^2 + z + 1\\
  \Phi_4(z) &= z^2 + 1\\
  \Phi_5(z) &= z^4 + z^3 + z^2 + z + 1\\
  \Phi_6(z) &= z^2 - z + 1
}
\end{frame}


\begin{frame}
\frametitle{Using $\Phi_n$ to create a field with $n$th roots of unity}
We will begin with the ring
Our goal is to create the ring
\aln{
  \mathbb{K}[x_1, \ldots, x_n]
}
where $\mathbb{K}$ is a field containing the primitive $n$th roots of unity. We can create such a field as follows:
\aln{
  \mathbb{K} = \frac{\mathbb{Q}[z]}{\Phi_n(z)}.
}
Using the relationship
\aln{
  \left( \frac{\mathbb{Q}[z]}{\Phi_n(z)} \right)[x_1, \ldots, x_n] \cong \frac{\mathbb{Q}[x_1, \ldots, x_n, z]}{\Phi_n},
}
We can choose
\aln{
  \mathbb{K}[x_1, \ldots, x_n] = \frac{\mathbb{Q}[x_1, \ldots, x_n, z]}{\Phi_n(z)}.
}
\end{frame}

\subsection{Implementation}

\begin{frame}[fragile]
\frametitle{Implementation using Macaulay2}
\begin{verbatim}
    R = QQ[x_1, x_2, x_3, x_4, x_5, x_6, z];
    CYCLOTOMIC_POLY = z^2 - z + 1;
    R = R/CYCLOTOMIC_POLY;
    IDEAL_GEN = {x_1^6 - 1, x_2^6 - 1, x_3^6 - 1,
             x_4^6 - 1, x_5^6 - 1, x_6^6 - 1,
	     z*x_1 - x_2,
	     z*x_2 - x_3,
	     z*x_3 - x_4,
	     z*x_4 - x_5,
	     z*x_5 - x_6,
	     z*x_6 - x_1};
    idealOfGraph = ideal IDEAL_GEN;
    graphBasis = flatten entries gens gb idealOfGraph;
    print toString graphBasis;
\end{verbatim}
\end{frame}

\section{Mathematical Result}

\begin{frame}
\frametitle{Using these results to prove a conjecture}
We want to prove the following conjecture:
\begin{conj}
	For a graph $G$ with $n$ vertices and one or more Hamiltonian cycles, $x_n^n - 1$ is in the reduced Gr\"obner basis of $H_G$.
\end{conj}
This is true for graphs with a single cycle, from a previous theorem.
\end{frame}

\subsection{Algebraic Tools}

\begin{frame}
\frametitle{Extending to Graphs with multiple Hamiltonian cycles}
We will use the following theorem to relate the encoding of a single cycle to multiple cycles:
\begin{thm}
Let $G$ be a connected directed graph with $n$ vertices. Then,
\begin{align*}
	H_G = \bigcap_C H_{G,C},
\end{align*}
where $C$ ranges over all Hamiltonian cycles of the graph $G$.
\end{thm}
\end{frame}

\begin{frame}
\frametitle{More work to do}
While this theorem reveals a potential strategy for proving the conjecture, computing the Gr\"obner basis of intersecting ideals is still very complicated.\\
\vfill
Can we reduce the problem further?
\end{frame}

\begin{frame}
\frametitle{Elimination Theory}
One way to reduce the scope of this problem is to examine an \emph{elimination ideal}, defined as follows:
\begin{defin}
Given $I = \langle f_1, \ldots, f_s \rangle \subseteq k[x_1, \ldots, x_n]$ the $\ell$-th \textbf{elimination ideal} $I_\ell$ is the ideal of $k[x_{\ell+1}, \ldots, x_n]$ defined by
\begin{align*}
	I_\ell = I \cap k[x_{\ell+1}, \ldots, x_n].
\end{align*}
\end{defin}
We would like to show that $H_G \cap k[x_n] = \langle x_n^n - 1 \rangle$ for any graph $G$ with one or more Hamiltonian cycles.\\
Can Gr\"obner bases help us do this?
\end{frame}

\begin{frame}
\frametitle{Elimination Theory}
Fortunately, they can. We can use the Elimination Theorem:
\begin{thm}
Let $I \subseteq k[x_1, \ldots, x_n]$ be an ideal and let $G$ be a Gr\"obner basis of $I$ with respect to lex order where $x_1 > x_2 > \cdots > x_n$. Then, for every $0 < \ell < n$, the set
\begin{align*}
	G_\ell = G \cap k[x_{\ell+1}, \ldots, x_n]
\end{align*}
is a Gr\"obner basis of the $\ell$-th elimination ideal $I_\ell$.
\end{thm}
Source: Cox, Little, O'Shea
\end{frame}

\subsection{Proof Strategy}

\begin{frame}
\frametitle{Example with two Hamiltonian cycles}
Let $G$ be a graph with $n$ vertices and two Hamiltonian cycles, represented by the cycle ideals $H_{G,C_1}$ and $H_{G,C_2}$. Then we find
\begin{align*}
	H_G \cap k[x_n] &= (H_{G,C_1} \cap H_{G,C_2}) \cap k[x_n]\\
	&= (H_{G,C_1} \cap k[x_n]) \cap (H_{G,C_2} \cap k[x_n])\\
	&= \langle x_n^n - 1 \rangle \cap \langle x_n^n - 1 \rangle\\
	&= \langle x_n^n - 1 \rangle.
\end{align*}
Therefore, the polynomial $x_n^n - 1$ is the sole generator of the polynomials in $H_G$ that contain $x_n$ alone.
\end{frame}

\section{Conclusion}

\begin{frame}
\frametitle{Future work}
A number of conjectures from this summer still remain. For instance, we present the following conjecture:
\begin{conj}
	Let $G$ be a graph with $n$ vertices, and $k \geq 2$. There exists $g$ in the reduced Gr\"obner basis of $H_G$ such that $\mathrm{LT}(g) = x_i^k$ for some $i$ such that $1 \leq i < n$ if and only if $G$ has more than one Hamiltonian cycle.
\end{conj}
\end{frame}

\begin{frame}
\frametitle{Potential applications}
Hopefully, by proving facts about the structure of graph ideals, we can learn more about graphs themselves. One such application would be to prove or disprove Sheehan's Conjecture:
\begin{conj}
  Every 4-regular graph has at least two Hamiltonian cycles.
\end{conj}
\end{frame}



\end{document}
